%!TEX root = main.tex
\chapter{Summary (Danish)}
\begin{otherlanguage}{danish}

Da man ofte deler smag med ens venner kan anbefalinger om bl.a. produkter forbedres hvis man kender brugeres sociale relationer.
Denne afhandling kigger på, i hvor høj grad man kan udlede et fremtidigt møde (som substitut for social relation) mellem to personer udfra geografiske og tidsmæssige sammenfald.
Data kommer fra Sonys mobil app Lifelog, indsamlet fra henholdsvis Sony medarbejdere henover tre måneder og rigtige brugere fra en enkelt måned.
Gennem data analyse, laver vi flere datasæt som vi træner med machine learning modeller.
Vi benytter en Logistic Regression baseline model med én feature og sammenligner med en Random Forest model trænet med flere features ud fra spatiotemporale data, bruger-alder og køn samt mobil app forbrug. Vi fandt at til udledning af et fremtidigt møde var der ikke en større forskel mellem baseline og Random Forest modellen for de fleste af datasættene, vi fandt desuden at modellerne havde højere ROC AUC score samt Precision og Recall for den positive klasse, at de mødes, med datasættet for rigtige brugere end Sony medarbejdere.

\end{otherlanguage}