%!TEX root = main.tex
\chapter{Summary (English)}
As you often like similar things, as your friends do, recommendations can be improved if you know if two people share social ties.
This thesis attempts to answer to what degree a future co-occurrence between two people (as a substitute for social ties) can be inferred, based on the users past spatiotemporal traces. 
Data is obtained from Sony Lifelog, a tracking mobile phone application, collected from Sony employees over three months and real-world users from a single month.
Through analysis of data we extract several datasets which we train using machine learning models.
We use a Logistic Regression baseline model trained with a single feature and compare it with a Random Forest model trained using numerous features based on spatiotemporal data, user age and gender data as well as app usage data. We found that for inferrence of a future co-occurrence there was not a large significant difference between the models for most of the datasets, furthermore we found that the models had a higher ROC AUC score as well as precision and recall for the positive class, a future co-occurrence using the dataset for real-world users than sony employees.