%!TEX root = main.tex
\chapter{Conclusion}
\label{chap:conclusion}
We built several datasets with a number of features based primarily on users spatiotemporal behavior collected from mobile sensors. Using these datasets we trained two classifiers and show it is possible to a certain extent to predict whether users meet in a future period as a proxy for social ties based on their patterns in a previous period. For our datasets we found no significant difference between the performance of a baseline logistic regression model with a single feature of two unique co-occurrences versus a random forest model trained on several features with respect to predicting the positive feature, did meet.

We found increasing performance in the models when using only a subset of active users and also an increase in performance using a dataset of diverse real-world users contrasted with a dataset of Sony employees only. Furthermore for the dataset of real-world users using just a subset of active users, we found mutual co-occurrences, specificity and co-occurrences weighted by number of people present to be most important for prediction.

Where as in other works, where the goal is predicting actual social ties, our features for prediction might work. We found in our work, where we used a co-occurrence in a next period as a proxy for social ties, that the features used were not significantly better than a baseline feature for prediction, except for the refined real-world dataset.

Svar på introduction
 
Perspective

