%!TEX root = main.tex
\chapter{Conclusion}
\label{chap:conclusion}
We built several datasets with a number of features based primarily on users spatiotemporal behavior, collected from mobile sensors. Using these datasets we trained two classifiers and showed, it is possible to a certain extent to predict whether users meet in a future period, as a proxy for social ties, based on their patterns in a previous period. For our datasets we found no significant difference between the performance of a baseline logistic regression model with a single feature of two unique co-occurrences, versus a random forest model trained on several features with respect to predicting the positive feature, did meet, except for the refined dataset with regular users.

We found increasing performance in the models when using only a subset of active users and also an increase in performance using a dataset of diverse regular users, contrasted with a dataset of Sony employees only. Furthermore for the dataset of real-world users using just a subset of active users, we found mutual co-occurrences, specificity and co-occurrences weighted by number of people present to be most important for prediction.

If the goal is to predict actual social ties, our features for prediction might have made a more significant difference, for our work however, they did not.

The inference of social ties or even just a co-occurrence, has implications for user privacy as analysis from past data shared by the users, can reveal information about future occurrences between them. This is information which users might not give permission to share, if they were aware of what can be inferred.

The benefits of inferring social ties are numerous as stated in \autoref{chap:Introduction}, however benefits from successful inference of just a co-occurrence can, if combined with possible information of where, provide users with quick directions or time-to-arrival information.

Future work could be spent on looking at the links between the users in the datasets, to assess who have had a co-occurrence with whom and how interconnected they are. We could also look at how well our features perform in predicting actual social ties between users.