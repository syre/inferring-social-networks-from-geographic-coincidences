%!TEX root = main.tex
\chapter{Discussion}
\label{chap:discussion}
In this section we will discuss the work we have done and the limitations of it.

We found that compared to the test dataset, our models worked better for production data in terms of predicting the positive class according to our $F_1$ scores for the models. We suspect this could be due to a low number of data-points for the users of the test-data. The homogeneity of the users as Sony employees could also be a factor as they are colleagues and mostly see each other as such. This means they have regular noisy co-occurrences which could be attributed to their work relations and not due to them being friends.

Furthermore a dataset with a subset of users which met our criteria of cross-occupancy worked best for training our models. This results in a higher probability of detecting whether user dyads had a co-occurrence or not in a period and also resulted in features generated on the basis of more data. 

We found that the performance of our random forest was not very significant compared to the baseline model at least in terms of predicting the positive class, it seems it does a better job of predicting the negative class based on the higher scores of the ROC AUC and confusion matrices produced, which we did not include in the report which takes into account true negative. The absence of a performance difference between the models could be due to the label of predicting whether they meet in the next period or not. The optimal features for prediction of a meeting in the next period might not align with the optimal features for predicting a social tie, thus two unique co-occurrences can be a sufficient metric. 

Initially we tried to infer social ties based on app usage for use as ground truth. We did not have directed call-logs available so we could not reliably say if for example two people were calling each other. However the idea was that if we could find users with a similar time stamp and duration in an app, they could perhaps be communicating with each other either through phone calls or messaging apps. After finding user-pairs that met that criteria we try to validate them looking at their co-occurrences, were they many or few, were they off Sony properties or on, a lot of co-occurrences off Sony properties could perhaps indicate a friendship. We identified the Android activity application name and package name for when a user is in a phone call called "Phone" and "package.com.android.incallui" respectively. Furthermore we identified a large number of messaging apps. We tried finding user pairs having activity in the phone app with a start time difference and end time difference of 10 seconds from each other. We looked at co-occurrences between user-pairs found this way and found that they either had a low number of co-occurrences around Sony properties or none at all.

Looking at more features related to app-usage could perhaps have been valuable, furthermore the Jaccard index similarity in app usage between users does not take into account disregarding the commonly used apps of the population of users, it could benefit from looking at if a user pair used apps uncommon for the general population of users.

The techniques used for binning also have a number of limitations.
We chose a relatively high resolution for spatial bins at 0.001 decimal degrees thinking a higher resolution provides more detail. A bin size of 1 hour on the hour was chosen for the time bins based on the assumption that 1 hour is a commonly occurring time period measurement for humans with respect to appointments. Different choices of time bin and spatial bin for binning could have been analyzed.

The method of time-binning used was a hourly bin instead of the time difference between the actual timestamps. This is limiting in the way that if two users occur in the same spatial bin but at the end and start of different hours, e.g. 11:59 and 12:01, it would not count as a co-occurrence.

Because of the figure of the earth being an oblate spheroid, the area of our spatial cells based on latitude and longitude degrees will vary significantly dependent on the location on the surface. The area of the cells will decrease as we traverse the globe from the equator to the poles. Equal-area partitioning could have been used for preserving the area measure.

The division in periods for a single month results in each period lasting 9-11 days. We chose working with a single month as users. This period interval might not be sufficient as it is not as natural a period measure for humans as for example a month or a week. Thus we do not capture the co-occurrences of friends who meet with friends at monthly intervals and only capture part of the weekly intervals.

The inference of social ties or even just a co-occurrence has implications for user privacy because based on analysis from data from the past shared by the users, we can to a degree gain information about future occurrences between them which they might not give permission to share if they were aware of what can be inferred.

We found a greater ratio of did meets for the test data, we suspect this can be due to places Sony employees meet in a work setting to a lesser degree than the already found places of exclusion. A larger radius of exclusion for the three Sony business-related properties could also have been tried.
Surprisingly the same is true for production data, although to a much lesser degree, while we ponder if this could be due to a 'contagion' effect on app usage between friends, we do not know.

We did not include altitude as a factor in calculating co-occurrences, two users with vastly different altitude levels but same spatial and time bin would thus co-occur independently of altitude. We disregarded altitude because we were unsure of how binning should work for altitude levels and varying altitude levels would often mean being inside buildings or below ground resulting in an inaccurate GPS signal.

Our filtering on accuracy at 55m makes the naive assumption that the user is in the center of the spatial bin with each side measuring 110m and thus remains inside even with their position being displaced 55m. A better solution could be filtering accuracy on the shortest distance to the actual cell perimeter for each location.

Due to time constraints running the code which took considerable time and having to travel to Sweden to run our code, we unfortunately did not get to implement all our features for the production data in Apache Spark.