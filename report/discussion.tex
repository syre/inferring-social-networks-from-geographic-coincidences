%!TEX root = main.tex
\chapter{Discussion}
\label{chap:discussion}
In this section we will discuss the work we have done and the limitations of it.

Different choices of time bin and spatial bin for binning could have been analyzed.

Because of the figure of the earth being an oblate spherioid, the area of our spatial cells based on latitude and longitude degrees will vary significantly dependent on the location on the surface. The area of the cells will decrease as we traverse the globe from the equator to the poles. Equal-area partitioning like HEALPix or Quad Sphere could have been used for preserving the area measure.

The pool of users from which the data is extracted is very homogenous as it only consists of Sony employees. Future work could use a more diverse pool of users as they might exhibit different behaviours in terms of mobility and in general.

As we did not have ground truth for friendship we used the metric if people met in the next period as a proxy. It is not known how well this metric conveys friendship and our work is thus limited in that regard.

The method of timebinning we used was a hourly bin instead of the time difference between the actual timestamps. This is limiting in the way that if two users occur in the same spatial bin but at the end and start of different hours, e.g. 11:59 and 12:01, it would not count as a co-occurrence.

Looking at more features related to app-usage could perhaps have been valuable, furthermore the Jaccard index similarity in app usage between users does not take into account disregarding the commonly used apps of the population of users, it could benefit from looking at if a user pair used apps uncommon for the general population of users.

\textcolor{red}{INDSÆT NOGET OM PRIVACY}

\textcolor{red}{Indsæt mulige grunde til at der er så stor ratio af did meets ifht. did not meets i vores fundne datasæt}

\textcolor{red}{The conclusion is that individuals who choose to reveal small amounts of public information about the times and locations of their activities may be inadvertently sending strong signals about certain of their social ties as well. [p. 5, c. 1]}


Altitude har også nogle negative værdier. Her er negative værdier sådan set ikke ugyldige, men dog sjældne og unormale. Hvis vi kigger på antal lokations opdateringer der har en negativ altitude på mindre end -50,000 (-50m), er der 9. Ved mindre end -10,000 (-10m) er der 14.

Periode-opdelingen virker bedst ved opdeling i måneder kontra per 10. dag f.eks. da måneder af mennesker bruges som en naturlig målestok.

\textcolor{red}{Tekst om accuracy, måske fejlagtig filtrering}