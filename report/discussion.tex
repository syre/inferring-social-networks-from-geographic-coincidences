%!TEX root = main.tex
\chapter{Discussion}
\label{chap:discussion}
In this section we will discuss the work we have done and the limitations of it.

We found that compared to the test dataset, our models worked better for production data in terms of predicting the positive class according to our $F_1$ scores for the models. We suspect this could be due to a low number of data-points for the users of the test-data. The homogeneity of the users as Sony employees could also be a factor as they are colleagues and mostly see each other as such. This means they have regular noisy co-occurrences which could be attributed to their work relations and not due to them being friends.

Furthermore a dataset with a subset of users which met our criteria of cross-occupancy worked best for training our models. This results in a higher probability of detecting whether user dyads had a co-occurrence or not in a period and also resulted in features generated on the basis of more data. We found that the performance of our random forest was not very significant compared to the baseline model at least in terms of predicting the positive class.

Initially we tried to infer friendship based on app usage. We did not have directed call-logs available so we could not reliably say if for example two people were calling each other. However the idea was that if we could find users with a similar time stamp and duration in an app, they could perhaps be communicating with each other either through phone calls or messaging apps. After finding user-pairs that met that criteria we try to validate them looking at their co-occurrences, were they many or few, were they off Sony properties or on, a lot of co-occurrences off Sony properties could perhaps indicate a friendship. We identified the Android activity application name and package name for when a user is in a phone call called "Phone" and "package.com.android.incallui" respectively. Furthermore we identified a large number of messaging apps. We tried finding user pairs having activity in the phone app with a start time difference and end time difference of 10 seconds from each other. We looked at co-occurrences between user-pairs found this way and found that they either had a low number of co-occurrences around Sony properties or none at all.

We chose a relatively high resolution for spatial bins at 0.001 decimal degrees, however different choices of time bin and spatial bin for binning could have been analyzed.

Because of the figure of the earth being an oblate spheroid, the area of our spatial cells based on latitude and longitude degrees will vary significantly dependent on the location on the surface. The area of the cells will decrease as we traverse the globe from the equator to the poles. Equal-area partitioning could have been used for preserving the area measure.

The pool of users from which the test data is extracted and analyzed is very homogeneous as it only consists of Sony employees. Future work could use a more diverse pool of users as they might exhibit different behaviors in terms of mobility and in general.

As we did not have ground truth for friendship we used the metric if people met in the next period as a proxy. It is not known how well this metric conveys friendship and our work is thus limited in that regard.

The method of time-binning we used was a hourly bin instead of the time difference between the actual timestamps. This is limiting in the way that if two users occur in the same spatial bin but at the end and start of different hours, e.g. 11:59 and 12:01, it would not count as a co-occurrence.

Looking at more features related to app-usage could perhaps have been valuable, furthermore the Jaccard index similarity in app usage between users does not take into account disregarding the commonly used apps of the population of users, it could benefit from looking at if a user pair used apps uncommon for the general population of users.

The division in periods for a single month results in each period lasting 9-11 days. This period interval might not be sufficient as it is not as natural a period measure for humans as for example a month or a week. Thus we might not capture the co-occurrences of friends who meet with friends at monthly or weekly intervals.

Could have used users from "just" period 1 and 2 and users of period 2 and 3.

Oversampling og Undersampling

\textcolor{red}{INDSÆT NOGET OM PRIVACY}

\textcolor{red}{Indsæt mulige grunde til at der er så stor ratio af did meets ifht. did not meets i vores fundne datasæt, for test data kan det være andre steder sony medarbejdere har mødtes vi ikke har taget højde for eller måske mødes de bare udover arbejde?, for prod data ved vi det reelt ikke, hvis det er små byer kan de møde hinanden ofte måske}


Altitude har også nogle negative værdier. Her er negative værdier sådan set ikke ugyldige, men dog sjældne og unormale. Hvis vi kigger på antal lokations opdateringer der har en negativ altitude på mindre end -50,000 (-50m), er der 9. Ved mindre end -10,000 (-10m) er der 14.



\textcolor{red}{Tekst om accuracy, måske fejlagtig filtrering}

Due to time constraints running the code which took considerable time and having to travel to Sweden to run our code, we unfortunately did not get to implement all our features for the production data in Apache Spark.