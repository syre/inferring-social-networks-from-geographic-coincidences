%!TEX root = main.tex
\section{Introduction}
In recent years the popularity of sensor-driven mobile devices and the emergence of the big data revolution has facilitated the creation and analysis of large-scale datasets generated from different aspects of our daily lives\cite{lazer2009life}. The GPS and its related technologies have made it easy to trace the movements of people over time.

\subsection{Sub-introduction}
bla bla


\section{Literature review}
The problem of analyzing location traces and predicting social ties and friendship have been investigated in a number of works. Crandall et al\cite{ISTfGC} used spatiotemporal data embedded in images (geotagging) by users of Flickr and partitioned the globe into grid-like cells and time bins at different resolutions. They found that even a small number of co-occurrences was a predictor of social ties.



\textbf{Result and methods}\\
We have partitioned the globe into grid-like cells with a cell resolution of .001\degree, and 1-hour time resolution. 

\textcolor{red}{If two persons have been on a few distinct cells within the time-period \texttt{t}, it can indicate that these two persons are directly tied in the network. [p. 1, c. 1]}

\textit{Framework}\\
\textcolor{red}{We divide the geographic into cells of the size sxs. 
If two persons (\texttt{A \& B}) tagging a picture within the same time-period in the same cell \texttt{c}, we say the they hace a co-occurrence. If so, we count the number of distinct cells (\texttt{k}) [p. 1, c. 2]}


\textcolor{red}{The social tie increase strongly when \texttt{k} increase and \texttt{t} decrease. [p. 1, c. 2]
Big difference on the probability and the baseline [p. 2, c. 1]}

\textcolor{red}{The probability depends on the size of \texttt{s} but also depennds on the geographically enviroment (city; densely populated, or country-land) [p. 2, c. 1]}


\textbf{Discussion and conclusion}\\
\textcolor{red}{The conclusion is that individuals who choose to reveal small amounts of public information about the times and locations of their activities may be inadvertently sending strong signals about certain of their social ties as well. [p. 5, c. 1]}