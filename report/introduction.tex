%!TEX root = main.tex
\chapter{Introduction}
\label{chap:Introduction}

The analysis of human behavior and networks is a significant part in several fields of social science like psychology, anthropology and sociology. Before the advent of sensing devices, the analysis of behaviors and networks were limited due to the biases inherent in the data collection methods of the studies and the limit in the number of participants of the studies. Data collection methods like participant observation can result in several biases, amongst others an unwanted influence in the behavior of the participants by the presence of the researcher\cite{rosenthal1966experimenter}. Furthermore studies relying on self-reporting can be prone to recall errors by the participants\cite{stone1999science}\cite{bernard1980informant}.

In recent years the popularity of sensor-driven mobile devices like phones, tablets and wearables and the emergence of the big data revolution has facilitated the creation and analysis of large-scale datasets generated from monitoring different aspects of our daily lives\cite{lazer2009life}. This method of data-collection can alleviate the problems of the former methods, by removing the inaccuracy of self-reporting, the bias of unwanted influence and at the same time allows for massive scaling of the number of participants. Furthermore the research on social networks has witnessed a substantial increase from the popularity and growth of online social networks such as Facebook, Twitter and LinkedIn\cite{social_networks}. The mobile phone and its related technologies have made it easier to trace the movements of people over time and have resulted in numerous advancements in computational social science. It has been shown that individuals show significant regularity in their movement over time\cite{gonzalez2008understanding} and that their movement can be predicted with great probability independent of distance\cite{song2010limits}.The research on human mobility is important because of its wide range of applications in different areas. Understanding human mobility can better our understanding of how diseases such as malaria spread\cite{wesolowski2012quantifying}, if people have social ties\cite{crandall2010inferring}, predicting crime hotspots\cite{bogomolov2014once} amongst others.

In this thesis we will try to predict social ties, using a future co-occurrence as proxy, of users based on data collected from an activity tracker app called Sony Lifelog\cite{sonyLifeLog}. The prediction of a future co-occurrence is used, as the dataset lacks ground truth. We acknowledge a future co-occurrence between users cannot be equated as having social ties, as the co-occurrence can merely be coincidental. Even if the co-occurrence is not coincidental and they know each other, it might be on a superficial level.

Sony Lifelog tracks location data and app usage of the users and allows them to gain insights based on these data. The data is collected for a 3-month period from Sony employees from all over the world.
The inference of social ties have numerous applications for different problems. It can aid friendship suggestions for social network sites like Facebook and Twitter. If we can infer two people have social ties they might share common interests which can improve target advertising or recommendations\cite{yu2015investigating}.

We will try to predict social ties using several different features from several datasets, one exclusively from Sony employees and one from real-world users. The datasets are then used to train state of the art machine learning models of which a comparison will be made on their performance. Each feature will be examined to asses how relevant they are for prediction of social ties in our dataset. Furthermore we will include features based on homophily\cite{mcpherson2001birds} from user gender and age, as well as app-usage.

The thesis is organized as follows. In \autoref{chap:literature_review}, we will describe the relevant research conducted on the inference of social ties from human spatiotemporal data. In \autoref{chap:dataset} we will analyze and describe the dataset and try to gain insights we can use in our work. \autoref{chap:methods} will describe the methods used on the dataset and \autoref{chap:results} will describe the results we obtained. In \autoref{chap:discussion} we will discuss the work we have done, the problems and shortcomings, as well as the qualities. In \autoref{chap:conclusion} we conclude on the work and offer our suggestions for what could be done in future work.


