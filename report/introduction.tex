%!TEX root = main.tex
\section{Introduction}
The analysis of human behaviour and networks is a significant part in several fields of social science like psychology, anthropology and sociology. Before the advent of sensing devices, the analysis of behaviours and networks were limited due to the biases inherent in the data collection methods of the studies and the limit in the number of participants of the studies. Data collection methods like participant observation can result in an unwanted influence in the behaviour of the participants by the presence of the researcher, what is called the observer-expectancy effect.\textcolor{red}{citation} Furthermore studies relying on self-reporting can be prone to recall errors by the participants.\textcolor{red}{citation}.

In recent years the popularity of sensor-driven mobile devices like phones, tablets and wearables and the emergence of the big data revolution has facilitated the creation and analysis of large-scale datasets generated from monitoring different aspects of our daily lives\cite{lazer2009life}. This method of data-collection alleviates the problems of the former methods by removing the inaccuracy of self-reporting, the bias of observer-expectancy and allows massive scaling of the number of participants. Furthermore the research on social networks has witnessed a substantial increase from the popularity and growth of online social networks such as Facebook, Twitter and LinkedIn\cite{social_networks}. The mobile phone and its related technologies have made it easier to trace the movements of people over time and have resulted in numerous advancements in computational social science. It has been shown that individuals show significant regularity in their movement over time\cite{Uihmp} and that their movement can be predicted with great probability independent of distance\cite{LoPiHM}. In this thesis we will try to predict the social ties of users based on data collected from an activity tracker app called Sony Lifelog\cite{sonyLifeLog}. Sony Lifelog tracks location data and app usage of the users and allows them to gain insights based on these data. The data is collected from Sony employees from all over the world.
The inferrence of social ties have numerous applications for different problems. It can improve the accuracy of friendship suggestions for social network sites like Facebook and Twitter\textcolor{red}{citation}. It can better our understanding of the spreading of information or diseases\textcolor{red}{citation}. If we can infer two people have social ties they might share common interests which can improve targetted advertising\textcolor{red}{citation}.

We will try to describe the relevant research conducted on the inferrence of social ties from human spatiotemporal data. We will try to predict social ties using several different features from a refined dataset extracted from time periods of the raw dataset. The refined dataset is then used to train several binary classification machine learning algorithms. Each feature will be examined to asess how relevant they are for prediction of social ties in our dataset. Furthermore we will include features based on homophily and app-usage extracted from the data.

