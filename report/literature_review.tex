%!TEX root = main.tex
\chapter{Literature review}
\label{chap:literature_review}
The problem of analyzing location traces and predicting social ties and friendship have been investigated in a number of works.
Below we will discuss the relevant literature in the field.

Crandall et al.\cite{crandall2010inferring} used spatiotemporal data embedded in images (geotagging) by users of Flickr and partitioned the globe into grid-like cells and time bins at different resolutions. They found that even a small number of co-occurrences from geotags was a predictor of social ties.

Eagle et al.\cite{eagle} collected call-logs, cell-tower information and bluetooth data from mobile phones given to 94 participants and looked at differences in self-reported and behavioural data, as well as relationship inference. They found they were able to predict 96\% and 95\% of reciprocal friends and reciprocal nonfriends respectively, based on a factor 'extra-role' found from measurements of locations, communication and of behaviour characterized of being outside the work environment.

Cranshaw et al\cite{cranshaw2010bridging} looked at possible similarities between online social networks and offline co-location networks for the same users and tried to predict social ties in the social network, based on the offline co-locations using several features focusing on user mobility and co-location. They found location entropy to be valuable in prediction of social ties, as locations with high entropy makes chance encounters more probable and furthermore found, that features from user mobility and co-location outperformed predictions based on number co-location observations alone.

Cho et al.\cite{FaMUMiLBSN} used data from location-based social networks Gowalla and Brightkite, and data from cell phone towers from 2 million users and found that, long-distance travels are more influenced by social ties than short-distance travels, thus people are more likely to travel near a friend when traveling far. Furthermore they found similarity in movement trajectories, is a strong indication of a tie in a social network.

Pham et al. in a number of works looked at inferring social ties from spatiotemporal co-occurrences\cite{pham2011towards}\cite{iRWRfSD}. The authors proposed a geo-social model (GEOSO) for capturing the relationship between co-occurrences and social ties. Inherent in the model, is the two properties that a higher number of common cells visited by two users, the socially closer they are (compatibility), and if two users repeatedly visits the same place together, they are more likely socially close (commitment).

Our work share similarities with these works but also differences.

The means of getting location are different from the works we have discussed. Some use direct measures of location updates like GPS\cite{cranshaw2010bridging} while others use periodic measures like cell-tower information\cite{eagle}, geotagged images\cite{crandall2010inferring} or check-ins from location-based social networks\cite{FaMUMiLBSN}.

Our work consists of location updates from GPS-enabled mobile phones which allows a greater resolution than cell-tower information and location-based social networks and with a higher update frequency than geotagged images.

The means for getting ground truth data with friendship and social ties varies, while some use mobile phone call and messaging logs, others use self-report information\cite{eagle}, links from online social networks\cite{cranshaw2010bridging}\cite{crandall2010inferring} or proximity measures derived from Bluetooth\cite{eagle} or Wi-Fi range data\cite{IPISTWS}.

Our dataset does not consist of ground truth of social ties between users, therefore we will use a future co-occurrence as a proxy for a social tie.