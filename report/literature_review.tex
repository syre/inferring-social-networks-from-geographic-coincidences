%!TEX root = main.tex
\section{Literature review}
The problem of analyzing location traces and predicting social ties and friendship have been investigated in a number of works. Crandall et al.\cite{ISTfGC} used spatiotemporal data embedded in images (geotagging) by users of Flickr and partitioned the globe into grid-like cells and time bins at different resolutions. They found that even a small number of co-occurrences was a predictor of social ties. 
Eagle et al.\cite{eagle} investigated data collected from mobile phones given to 100 participants and looked at differences in self-reported and behavioural data as well as relationship inference. They found they were able to predict 96\% and 95\% of reciprocal friends and reciprocal nonfriends respectively based on a factor 'extra-role' found from measurements of locations, communication and of behaviour characterized of being outside the work environment. Cranshaw et al\cite{cranshaw2010bridging} looked at possible similarities between online social networks and offline co-location networks for the same users and tried to predict social ties in the social network based on the offline co-locations using several features focusing on user mobility and co-location. They found location entropy to be valuable in prediction of social ties as locations with high entropy makes chance encounters more probable and furthermore found that features from user mobility and co-location outperformed predictions based on number co-location observations alone.
