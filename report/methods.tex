%!TEX root = main.tex
\section{Methods}
In this section we will outline what methods we have used for conducting the work.
\subsection{Binning}
We have partitioned or binned the globe into a grid where the cells have a resolution of .001\degree, and 1-hour time resolution. In the following section the term grid cell or spatial bin will be used synonymously. Each grid cell has a unique ID, this is obtained by considering the grid as a matrix and flattening it to a list by traversing from left to right and top to bottom (row-first ordering) and using the list index as an ID. This makes a total possible spatial bin indexes of: $(190\cdot100)\cdot(360\cdot100)=68.400.000.000$ We calculated the time bins by counting every hour from the start of the day of the earliest date appearing in the dataset, which has the ISO 8601 timestring: "2015-08-09 22:25:33.766+02". The latest day has the ISO 8601 timestring: "2015-12-01 00:59:15.738+01". This leaves us with $2738$ possible time bin IDs. For generating our dataset we count co-occurrences between pairs only once per spatial and time bin.

\subsection{Features for the dataset}
\subsubsection{Diversity}
We use the diversity measure taken from the papers of Shahabi and Pham\cite{iRWRfSD}\cite{AEBMtISSfSD}.
Diversity is a measure of how important the spatial locations of the co-occurrences between a pair of persons are, given how many times they appear.
It exists in two forms, one using Shannon entropy another using Rényi entropy, we will use Shannon.
$$ H^S_{ij}=-\sum\limits_{l}P^l_{ij}logP^l_{ij}= -\sum\limits_{l,c_ij,l\neq 0}\frac{c_{ij,l}}{f_{ij}}log\frac{c_{ij,l}}{f_{ij}}$$

\textcolor{red}{If two persons have been on a few distinct cells within the time-period \texttt{t}, it can indicate that these two persons are directly tied in the network. [p. 1, c. 1]}

\textit{Framework}
\textcolor{red}{We divide the geographic into cells of the size sxs. 
If two persons (\texttt{A \& B}) tagging a picture within the same time-period in the same cell \texttt{c}, we say the they hace a co-occurrence. If so, we count the number of distinct cells (\texttt{k}) [p. 1, c. 2]}


\textcolor{red}{The social tie increase strongly when \texttt{k} increase and \texttt{t} decrease. [p. 1, c. 2]
Big difference on the probability and the baseline [p. 2, c. 1]}

\textcolor{red}{The probability depends on the size of \texttt{s} but also depennds on the geographically enviroment (city; densely populated, or country-land) [p. 2, c. 1]}