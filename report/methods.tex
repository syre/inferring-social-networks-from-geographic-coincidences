%!TEX root = main.tex
\section{Methods}
In this section we will outline what methods we have used for conducting the work.
\subsection{Data preperation}
\subsubsection{Binning}
We have partitioned or binned the globe into a grid where the cells have a resolution of .001\degree thus to 3 decimal places, and 1-hour time resolution. In the following section the term grid cell or spatial bin will be used synonymously. Each grid cell has a unique ID, this is obtained by considering the grid as a matrix and flattening it to a list by traversing from left to right and top to bottom (row-first ordering) and using the list index as an ID. This makes a total possible spatial bin indexes of: $(190\times dec)\times(360\times dec)=68.400.000.000$ with $dec=3$ We calculated the time bins by counting every hour from the start of the day of the earliest date appearing in the dataset, which has the ISO 8601 timestring: "2015-08-09 22:25:33.766+02". The latest day has the ISO 8601 timestring: "2015-12-01 00:59:15.738+01". This leaves us with $2738$ possible time bin IDs. For generating our dataset we count co-occurrences between pairs only once per spatial and time bin.

\subsection{Modeling}
In this subsection we will describe the model and features used for prediction.

We will try to answer the following question: Can we predict if people meet in a period based on their spatiotemporal patterns in a previous period.
The models and algorithms we use are implemented in \textit{scikit-learn}\cite{scikit-learn} a Python module for machine learning. 
First we will describe the different features of which our dataset is compromised.
\paragraph{Number of co-occurrences}
The total number of co-occurrences between users.
\paragraph{Diversity of a co-occurrence}
We use the diversity measure taken from the papers of Shahabi and Pham\cite{iRWRfSD}\cite{AEBMtISSfSD}.
Diversity is a measure of how important the spatial locations of the co-occurrences between a pair of persons are, given how many times they appear.
It exists in two forms, one using Shannon entropy another using Rényi entropy, we will use Shannon.
The Shannon entropy is defined by Equation \ref{eq:shannon_entropy}
\begin{equation}
\label{eq:shannon_entropy}
H^S_{ij}=-\sum\limits_{l}P^l_{ij} \log P^l_{ij}= -\sum\limits_{l,c_ij,l\neq 0}\frac{c_{ij,l}}{f_{ij}}\log \frac{c_{ij,l}}{f_{ij}}
\end{equation}
where $f_{ij}$ is the \textit{frequency}, the total number of co-occurrences between user $i$ and user $j$, and $c_{ij,l}$, the \textit{local frequency} is the total number of co-occurrences between user $i$ and $j$ at location $l$.
From this the diversity is defined by taking the exponential function of the entropy defined in Equation \ref{eq:diversity}:
\begin{equation}
\label{eq:diversity}
D^s_{ij} = exp(H^S_{ij})
\end{equation}

\paragraph{Weighted frequency}
We use the weighted frequency measure taken from the papers of Shababi and Pham\cite{iRWRfSD}\cite{AEBMtISSfSD}.
Weighted frequency is a measure of how important the co-occurrences at non-popular places are.
The weighed frequency is defined by Equation \ref{eq:weighted_frequency}
\begin{equation}
\label{eq:weighted_frequency}
F_{ij}=\sum\limits_{l}c_{ij,l} \times \exp(-H_l)
\end{equation}
where $H_l$ is the Location Entropy defined in Equation \ref{eq:location_entropy}
\begin{equation}
\label{eq:location_entropy}
H_l = \sum\limits_{u, P_{u,l}\neq0} P_{u,l}\log P_{u,l}
\end{equation}
\paragraph{Co-occurrences weighted with respect to each location}
We use the weighted co-occurrences measure taken from the master's thesis of P. Sapieżyński\cite{IMM2013-06556}.

\paragraph{Timely arrival and leaving}
We use the timely arrival and leaving measure taken from the master's thesis of P. Sapieżyński\cite{IMM2013-06556}.

\paragraph{Unique spatial bins}
The number of unique spatial bins between two users.

\paragraph{Countries in common}
The number of common countries which two users have visited.

\subsection{}